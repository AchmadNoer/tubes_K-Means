% Options for packages loaded elsewhere
\PassOptionsToPackage{unicode}{hyperref}
\PassOptionsToPackage{hyphens}{url}
%
\documentclass[
]{article}
\usepackage{lmodern}
\usepackage{amssymb,amsmath}
\usepackage{ifxetex,ifluatex}
\ifnum 0\ifxetex 1\fi\ifluatex 1\fi=0 % if pdftex
  \usepackage[T1]{fontenc}
  \usepackage[utf8]{inputenc}
  \usepackage{textcomp} % provide euro and other symbols
\else % if luatex or xetex
  \usepackage{unicode-math}
  \defaultfontfeatures{Scale=MatchLowercase}
  \defaultfontfeatures[\rmfamily]{Ligatures=TeX,Scale=1}
\fi
% Use upquote if available, for straight quotes in verbatim environments
\IfFileExists{upquote.sty}{\usepackage{upquote}}{}
\IfFileExists{microtype.sty}{% use microtype if available
  \usepackage[]{microtype}
  \UseMicrotypeSet[protrusion]{basicmath} % disable protrusion for tt fonts
}{}
\makeatletter
\@ifundefined{KOMAClassName}{% if non-KOMA class
  \IfFileExists{parskip.sty}{%
    \usepackage{parskip}
  }{% else
    \setlength{\parindent}{0pt}
    \setlength{\parskip}{6pt plus 2pt minus 1pt}}
}{% if KOMA class
  \KOMAoptions{parskip=half}}
\makeatother
\usepackage{xcolor}
\IfFileExists{xurl.sty}{\usepackage{xurl}}{} % add URL line breaks if available
\IfFileExists{bookmark.sty}{\usepackage{bookmark}}{\usepackage{hyperref}}
\hypersetup{
  pdftitle={Tugas Besar Pembelajaran Mesin - K-Means},
  pdfauthor={Dibuat oleh Aziz (17523001) dan Ihya (17523126)},
  hidelinks,
  pdfcreator={LaTeX via pandoc}}
\urlstyle{same} % disable monospaced font for URLs
\usepackage[margin=1in]{geometry}
\usepackage{color}
\usepackage{fancyvrb}
\newcommand{\VerbBar}{|}
\newcommand{\VERB}{\Verb[commandchars=\\\{\}]}
\DefineVerbatimEnvironment{Highlighting}{Verbatim}{commandchars=\\\{\}}
% Add ',fontsize=\small' for more characters per line
\usepackage{framed}
\definecolor{shadecolor}{RGB}{248,248,248}
\newenvironment{Shaded}{\begin{snugshade}}{\end{snugshade}}
\newcommand{\AlertTok}[1]{\textcolor[rgb]{0.94,0.16,0.16}{#1}}
\newcommand{\AnnotationTok}[1]{\textcolor[rgb]{0.56,0.35,0.01}{\textbf{\textit{#1}}}}
\newcommand{\AttributeTok}[1]{\textcolor[rgb]{0.77,0.63,0.00}{#1}}
\newcommand{\BaseNTok}[1]{\textcolor[rgb]{0.00,0.00,0.81}{#1}}
\newcommand{\BuiltInTok}[1]{#1}
\newcommand{\CharTok}[1]{\textcolor[rgb]{0.31,0.60,0.02}{#1}}
\newcommand{\CommentTok}[1]{\textcolor[rgb]{0.56,0.35,0.01}{\textit{#1}}}
\newcommand{\CommentVarTok}[1]{\textcolor[rgb]{0.56,0.35,0.01}{\textbf{\textit{#1}}}}
\newcommand{\ConstantTok}[1]{\textcolor[rgb]{0.00,0.00,0.00}{#1}}
\newcommand{\ControlFlowTok}[1]{\textcolor[rgb]{0.13,0.29,0.53}{\textbf{#1}}}
\newcommand{\DataTypeTok}[1]{\textcolor[rgb]{0.13,0.29,0.53}{#1}}
\newcommand{\DecValTok}[1]{\textcolor[rgb]{0.00,0.00,0.81}{#1}}
\newcommand{\DocumentationTok}[1]{\textcolor[rgb]{0.56,0.35,0.01}{\textbf{\textit{#1}}}}
\newcommand{\ErrorTok}[1]{\textcolor[rgb]{0.64,0.00,0.00}{\textbf{#1}}}
\newcommand{\ExtensionTok}[1]{#1}
\newcommand{\FloatTok}[1]{\textcolor[rgb]{0.00,0.00,0.81}{#1}}
\newcommand{\FunctionTok}[1]{\textcolor[rgb]{0.00,0.00,0.00}{#1}}
\newcommand{\ImportTok}[1]{#1}
\newcommand{\InformationTok}[1]{\textcolor[rgb]{0.56,0.35,0.01}{\textbf{\textit{#1}}}}
\newcommand{\KeywordTok}[1]{\textcolor[rgb]{0.13,0.29,0.53}{\textbf{#1}}}
\newcommand{\NormalTok}[1]{#1}
\newcommand{\OperatorTok}[1]{\textcolor[rgb]{0.81,0.36,0.00}{\textbf{#1}}}
\newcommand{\OtherTok}[1]{\textcolor[rgb]{0.56,0.35,0.01}{#1}}
\newcommand{\PreprocessorTok}[1]{\textcolor[rgb]{0.56,0.35,0.01}{\textit{#1}}}
\newcommand{\RegionMarkerTok}[1]{#1}
\newcommand{\SpecialCharTok}[1]{\textcolor[rgb]{0.00,0.00,0.00}{#1}}
\newcommand{\SpecialStringTok}[1]{\textcolor[rgb]{0.31,0.60,0.02}{#1}}
\newcommand{\StringTok}[1]{\textcolor[rgb]{0.31,0.60,0.02}{#1}}
\newcommand{\VariableTok}[1]{\textcolor[rgb]{0.00,0.00,0.00}{#1}}
\newcommand{\VerbatimStringTok}[1]{\textcolor[rgb]{0.31,0.60,0.02}{#1}}
\newcommand{\WarningTok}[1]{\textcolor[rgb]{0.56,0.35,0.01}{\textbf{\textit{#1}}}}
\usepackage{graphicx,grffile}
\makeatletter
\def\maxwidth{\ifdim\Gin@nat@width>\linewidth\linewidth\else\Gin@nat@width\fi}
\def\maxheight{\ifdim\Gin@nat@height>\textheight\textheight\else\Gin@nat@height\fi}
\makeatother
% Scale images if necessary, so that they will not overflow the page
% margins by default, and it is still possible to overwrite the defaults
% using explicit options in \includegraphics[width, height, ...]{}
\setkeys{Gin}{width=\maxwidth,height=\maxheight,keepaspectratio}
% Set default figure placement to htbp
\makeatletter
\def\fps@figure{htbp}
\makeatother
\setlength{\emergencystretch}{3em} % prevent overfull lines
\providecommand{\tightlist}{%
  \setlength{\itemsep}{0pt}\setlength{\parskip}{0pt}}
\setcounter{secnumdepth}{-\maxdimen} % remove section numbering

\title{Tugas Besar Pembelajaran Mesin - K-Means}
\author{Dibuat oleh Aziz (17523001) dan Ihya (17523126)}
\date{}

\begin{document}
\maketitle

Pada tugas ini kami akan menyelesaikan masalah clustering menggunakan
metode k-means pada dataset ikan. Akan dibahas juga mengenai
perbandingan antara data asli dengan hasil k-means melalui sebuah plot
dan juga temuan-temuan lainnya. Ditampilkan juga kode beserta outputnya.
Untuk yang ingin melihat kodenya secara utuh bisa cek di
\href{https://github.com/AchmadNoer/tubes_K-Means}{Github} berikut.

\hypertarget{data}{%
\paragraph{Data}\label{data}}

Data yang digunakan dalam kasus clustering ini adalah dataset ikan yang
didapatkan dari situs \href{https://www.kaggle.com/datasets}{kaggle.com}
dengan kata kunci
\href{https://www.kaggle.com/aungpyaeap/fish-market}{Fish Market}
berformat csv. Di dalamnya terdapat kolom spesies ikan dan berbagai
macam ukuran dalam satuan centimeter. Untuk memuatnya ke dalam RStudio
dapat menggunakan kode \texttt{read.csv()} seperti pada baris ke-1.

\begin{Shaded}
\begin{Highlighting}[numbers=left,,]
\NormalTok{dataIkan <-}\StringTok{ }\KeywordTok{read.csv}\NormalTok{(}\StringTok{"Ikan.csv"}\NormalTok{, }\DataTypeTok{stringsAsFactors=}\OtherTok{TRUE}\NormalTok{, }\DataTypeTok{fileEncoding=}\StringTok{"UTF-8-BOM"}\NormalTok{)}
\KeywordTok{head}\NormalTok{(dataIkan)}
\end{Highlighting}
\end{Shaded}

\begin{verbatim}
##   Species Weight Length1 Length2 Length3  Height  Width
## 1   Bream    242    23.2    25.4    30.0 11.5200 4.0200
## 2   Bream    290    24.0    26.3    31.2 12.4800 4.3056
## 3   Bream    340    23.9    26.5    31.1 12.3778 4.6961
## 4   Bream    363    26.3    29.0    33.5 12.7300 4.4555
## 5   Bream    430    26.5    29.0    34.0 12.4440 5.1340
## 6   Bream    450    26.8    29.7    34.7 13.6024 4.9274
\end{verbatim}

Dari dataset ikan yang digunakan untuk clustering sebagai variabelnya
akan dipilih 2 saja yaitu data \texttt{Width} dan \texttt{Height}.
Pemilihan parameter ini setelah didiskusikan, dirasa cukup sebagai
parameter yang mewakili dari masing-masing spesies ikan.

\begin{Shaded}
\begin{Highlighting}[numbers=left,,]
\NormalTok{dataIkan.ciri <-}\StringTok{ }\NormalTok{dataIkan[, }\DecValTok{6}\OperatorTok{:}\DecValTok{7}\NormalTok{]}
\NormalTok{dataIkan.jenis <-}\StringTok{ }\NormalTok{dataIkan[, }\StringTok{"Species"}\NormalTok{]}
\KeywordTok{set.seed}\NormalTok{(}\DecValTok{62}\NormalTok{)}
\NormalTok{hasilKMeans <-}\StringTok{ }\KeywordTok{kmeans}\NormalTok{(dataIkan.ciri, }\DecValTok{7}\NormalTok{)}
\end{Highlighting}
\end{Shaded}

Pada baris ke-4, telah ditentukan bahwa dari fungsi \texttt{kmeans} akan
mencari 7 cluster dari dataset ikan dengan 2 variabel yang sudah
ditentukan tadi. Tujuh disini maksudnya adalah 7 spesies ikan karena
sesuai dengan jumlah spesies ikan di dataset. Selanjutnya, hasil
clustering dari fungsi \texttt{kmeans} akan divisualisasikan sehingga
dapat dibandingkan dengan data aslinya.

Sebelum masuk ke plotting, ada tahap yang berfungsi untuk mengevaluasi
cost function atau yang disebut sebagai distortion function. Di sini
nilai distortion pada iterasi terakhir dapat dilihat melalui
\texttt{tot.withinss}.

\begin{Shaded}
\begin{Highlighting}[numbers=left,,]
\NormalTok{hasilKMeans}\OperatorTok{$}\NormalTok{tot.withinss}
\end{Highlighting}
\end{Shaded}

\begin{verbatim}
## [1] 145.1151
\end{verbatim}

\hypertarget{visualisasi-data}{%
\paragraph{Visualisasi Data}\label{visualisasi-data}}

\begin{Shaded}
\begin{Highlighting}[numbers=left,,]
\KeywordTok{par}\NormalTok{(}\DataTypeTok{mfrow=}\KeywordTok{c}\NormalTok{(}\DecValTok{1}\NormalTok{,}\DecValTok{2}\NormalTok{))}
\KeywordTok{plot}\NormalTok{(dataIkan.ciri, }\DataTypeTok{col =}\NormalTok{ hasilKMeans}\OperatorTok{$}\NormalTok{cluster, }\DataTypeTok{main=}\StringTok{"K-Means"}\NormalTok{)}
\KeywordTok{plot}\NormalTok{(dataIkan.ciri, }\DataTypeTok{col =}\NormalTok{ dataIkan.jenis, }\DataTypeTok{main=}\StringTok{"Original"}\NormalTok{)}
\end{Highlighting}
\end{Shaded}

\includegraphics{Tugas-Besar-K-Means_files/figure-latex/unnamed-chunk-4-1.pdf}

Plot sebelah kiri adalah hasil clustering menggunakan fungsi
\texttt{kmeans}, sedangkan plot sebelah kanan adalah plot dataset
spesies ikan berdasarkan dua variabel yaitu \texttt{Height} dan
\texttt{Width}.

Perhatikan dua plot diatas, dapat dilihat bahwa hasil clustering
merepresentasikan kelompok spesies dengan cukup bagus apabila
dibandingkan dengan plot dari data aslinya. Tujuh spesies
direpresentasikan dengan warna yang berbeda-beda. Walaupun urutan warna
yang tidak sama pada hasil clustering (plot kiri) dengan data aslinya
(plot kanan), hal tersebut tidak menjadi masalah.

Di bawah ini, kita akan memvisualisasikan hasil clustering di atas,
menggunakan package \texttt{ggplot2}.

Baris ke-4 hingga ke-6 adalah plotting menggunakan \texttt{ggplot2}.
Pada bagian ini akan diberikan efek radius atau jangkauan untuk
masing-masing clusternya.

\begin{Shaded}
\begin{Highlighting}[numbers=left,,]
\NormalTok{dataIkan.ciri}\OperatorTok{$}\NormalTok{cluster <-}\StringTok{ }\KeywordTok{factor}\NormalTok{(hasilKMeans}\OperatorTok{$}\NormalTok{cluster)}
\NormalTok{centers <-}\StringTok{ }\KeywordTok{as.data.frame}\NormalTok{(hasilKMeans}\OperatorTok{$}\NormalTok{centers)}

\KeywordTok{library}\NormalTok{(}\StringTok{"ggplot2"}\NormalTok{)}
\KeywordTok{ggplot}\NormalTok{() }\OperatorTok{+}\StringTok{ }
\StringTok{  }\KeywordTok{geom_point}\NormalTok{(}\DataTypeTok{data =}\NormalTok{ dataIkan.ciri, }\KeywordTok{aes}\NormalTok{(}\DataTypeTok{x =}\NormalTok{ Height, }\DataTypeTok{y =}\NormalTok{ Width, }\DataTypeTok{color =}\NormalTok{ cluster)) }\OperatorTok{+}\StringTok{ }
\StringTok{  }\KeywordTok{geom_point}\NormalTok{(}\DataTypeTok{data =}\NormalTok{ centers, }\KeywordTok{aes}\NormalTok{(}\DataTypeTok{x =}\NormalTok{ Height, }\DataTypeTok{y =}\NormalTok{ Width, }\DataTypeTok{color =} \StringTok{"Center"}\NormalTok{), }
             \DataTypeTok{size =} \DecValTok{30}\NormalTok{, }\DataTypeTok{alpha =} \FloatTok{0.2}\NormalTok{, }\DataTypeTok{show.legend =} \OtherTok{FALSE}\NormalTok{)}
\end{Highlighting}
\end{Shaded}

\includegraphics{Tugas-Besar-K-Means_files/figure-latex/unnamed-chunk-5-1.pdf}

\end{document}
